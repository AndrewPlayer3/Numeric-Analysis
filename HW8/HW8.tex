%%%%%%%%%%%%%%%%%%%%%%%%%%%%%%%%%%%%%%%%%%%%%%%%%%%%%%%%%%%%%%%%%%%%%%%%%%%%%%%%%%%%%%%
%%%%%%%%%%%%%%%%%%%%%%%%%%%%%%%%%%%%%%%%%%%%%%%%%%%%%%%%%%%%%%%%%%%%%%%%%%%%%%%%%%%%%%%
% 
% This top part of the document is called the 'preamble'.  Modify it with caution!
%
% The real document starts below where it says 'The main document starts here'.

\documentclass[12pt]{article}

\usepackage{amssymb,amsmath,amsthm}
\usepackage[top=1in, bottom=1in, left=1.25in, right=1.25in]{geometry}
\usepackage{fancyhdr}
\usepackage{graphicx}
\usepackage{enumerate}
\usepackage{verbatim}
\usepackage{listings}
\usepackage{xcolor}

% Comment the following line to use TeX's default font of Computer Modern.
\usepackage{times,txfonts}

\definecolor{codegreen}{rgb}{0,0.6,0}
\definecolor{codegray}{rgb}{0.5,0.5,0.5}
\definecolor{codepurple}{rgb}{0.58,0,0.82}
\definecolor{backcolour}{rgb}{0.95,0.95,0.92}

\lstdefinestyle{mystyle}{
    backgroundcolor=\color{backcolour},   
    commentstyle=\color{codegreen},
    keywordstyle=\color{magenta},
    numberstyle=\tiny\color{codegray},
    stringstyle=\color{codepurple},
    basicstyle=\ttfamily\footnotesize,
    breakatwhitespace=false,         
    breaklines=true,                 
    captionpos=b,                    
    keepspaces=true,                 
    numbers=left,                    
    numbersep=5pt,                  
    showspaces=false,                
    showstringspaces=false,
    showtabs=false,                  
    tabsize=2
}

\lstset{style=mystyle}

\newtheoremstyle{homework}% name of the style to be used
  {18pt}% measure of space to leave above the theorem. E.g.: 3pt
  {12pt}% measure of space to leave below the theorem. E.g.: 3pt
  {}% name of font to use in the body of the theorem
  {}% measure of space to indent
  {\bfseries}% name of head font
  {:}% punctuation between head and body
  {2ex}% space after theorem head; " " = normal interword space
  {}% Manually specify head
\theoremstyle{homework} 

% Set up an Exercise environment and a Solution label.
\newtheorem*{exercisecore}{\@currentlabel}
\newenvironment{exercise}[1]
{\def\@currentlabel{#1}\exercisecore}
{\endexercisecore}

\newcommand{\localhead}[1]{\par\smallskip\noindent\textbf{#1}\nobreak\\}%
\newcommand\solution{\localhead{Solution:}}



% \newcommand{includematlab}[1]{\verbatiminput{#1}}

%%%%%%%%%%%%%%%%%%%%%%%%%%%%%%%%%%%%%%%%%%%%%%%%%%%%%%%%%%%%%%%%%%%%%%%%
%
% Stuff for getting the name/document date/title across the header
\makeatletter
\RequirePackage{fancyhdr}
\pagestyle{fancy}
\fancyfoot[C]{\ifnum \value{page} > 1\relax\thepage\fi}
\fancyhead[L]{\ifx\@doclabel\@empty\else\@doclabel\fi}
\fancyhead[C]{\ifx\@docdate\@empty\else\@docdate\fi}
\fancyhead[R]{\ifx\@docauthor\@empty\else\@docauthor\fi}
\headheight 15pt

\def\doclabel#1{\gdef\@doclabel{#1}}
\doclabel{Use {\tt\textbackslash doclabel\{MY LABEL\}}.}
\def\docdate#1{\gdef\@docdate{#1}}
\docdate{Use {\tt\textbackslash docdate\{MY DATE\}}.}
\def\docauthor#1{\gdef\@docauthor{#1}}
\docauthor{Use {\tt\textbackslash docauthor\{MY NAME\}}.}
\makeatother

%% General formatting parameters
\parindent 0pt
\parskip 12pt plus 1pt

% Shortcuts for blackboard bold number sets (reals, integers, etc.)
\newcommand{\Reals}{\ensuremath{\mathbb R}}
\newcommand{\Nats}{\ensuremath{\mathbb N}}
\newcommand{\Ints}{\ensuremath{\mathbb Z}}
\newcommand{\Rats}{\ensuremath{\mathbb Q}}
\newcommand{\Cplx}{\ensuremath{\mathbb C}}
%% Some equivalents that some people may prefer.
\let\RR\Reals
\let\NN\Nats
\let\II\Ints
\let\CC\Cplx

%%%%%%%%%%%%%%%%%%%%%%%%%%%%%%%%%%%%%%%%%%%%%%%%%%%%%%%%%%%%%%%%%%%%%%%%%%%%%%%%%%%%%%%
%%%%%%%%%%%%%%%%%%%%%%%%%%%%%%%%%%%%%%%%%%%%%%%%%%%%%%%%%%%%%%%%%%%%%%%%%%%%%%%%%%%%%%%
% 
% The main document start here.

% The following commands set up the material that appears in the header.
\doclabel{Math 426: Homework 8}
\docauthor{Andrew Player}
\docdate{October 21, 2020}

\begin{document}
\begin{exercise}{Supplemental 1}
Finish your code from question 19 of the worksheet on implementing
LU decomposition with partial pivoting.  Then use this code together
with the course page \texttt{lsolve} and your own \texttt{usolve}
to solve $A\mathbf{x}=\mathbf{b}$ where
\[
A=\begin{pmatrix} 9 & 3 & 2 & 0 &7 \\
7 & 6 & 9 & 6 & 4\\
2 & 7 & 7 & 8 & 2 \\
0 & 9 & 7 & 2 & 2 \\
7 & 3 & 6 & 4 & 3
\end{pmatrix}
\]
and $b=[35,58,53,37,39]^T$.  For the record, the true solution is $x=[0,1,2,3,4]^T$.
\end{exercise}
Code:
\begin{lstlisting} [language = Matlab]
function [L, U, P] = LUNoPivot(A)
    
  % Inititialize Everything
  [m,n] = size(A);
  L = eye(size(A));
  P = [1:n];
  U = A;
  
  for j = 1:n-1
      
      % Get the pivot
      [a, b] = max(abs(U(j:n,j)));
      
      % Swap the rows of U, L, and P
      U([j,b+j-1],j:n)   = U([b+j-1,j],j:n);
      L([j,b+j-1],1:j-1) = L([b+j-1,j],1:j-1);
      P(1,[j,b+j-1])     = P(1,[b+j-1,j]);

      % Compute the U and L entries
      for i = j+1:n
          L(i,j) = U(i,j)/U(j,j);
          U(i,j:n) = U(i,j:n) - L(i,j)*U(j,j:n);
      end
  end
end
\end{lstlisting}
Test Output:
\begin{lstlisting} [language=Matlab]
>> [L, U, P] = LUPivot(A)

L =

    1.0000         0         0         0         0
          0    1.0000         0         0         0
    0.7778    0.4074    1.0000         0         0
    0.2222    0.7037    0.3548    1.0000         0
    0.7778    0.0741    0.8548   -0.1222    1.0000


U =

    9.0000    3.0000    2.0000         0    7.0000
          0    9.0000    7.0000    2.0000    2.0000
          0   -0.0000    4.5926    5.1852   -2.2593
          0         0         0    4.7527   -0.1613
          0         0   -0.0000         0   -0.6810


P =

      1     4     2     3     5

>> y = psolve(P, L, b);
>> x = usolve(U, y)

x =

    -0.0000    1.0000    2.0000    3.0000    4.0000
\end{lstlisting}

\begin{exercise}{Exercise 7.8}
\end{exercise}
v = [4 5 -6]
\newline
2-norm = $\sqrt{4^2+5^2+(-6)^2} = \sqrt{16+25+36} = \sqrt{77}$
\newline
1-norm = $|4|+|5|+|-6| = 15$
\newline
$\infty$-norm = max\{$|4|, |5|, |-6|$\} = 6

\end{document}