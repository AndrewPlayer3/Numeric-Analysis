%%%%%%%%%%%%%%%%%%%%%%%%%%%%%%%%%%%%%%%%%%%%%%%%%%%%%%%%%%%%%%%%%%%%%%%%%%%%%%%%%%%%%%%
%%%%%%%%%%%%%%%%%%%%%%%%%%%%%%%%%%%%%%%%%%%%%%%%%%%%%%%%%%%%%%%%%%%%%%%%%%%%%%%%%%%%%%%
% 
% This top part of the document is called the 'preamble'.  Modify it with caution!
%
% The real document starts below where it says 'The main document starts here'.

\documentclass[12pt]{article}

\usepackage{amssymb,amsmath,amsthm}
\usepackage[top=1in, bottom=1in, left=1.25in, right=1.25in]{geometry}
\usepackage{fancyhdr}
\usepackage{graphicx}
\usepackage{enumerate}
\usepackage{verbatim}

% Comment the following line to use TeX's default font of Computer Modern.
\usepackage{times,txfonts}

\newtheoremstyle{homework}% name of the style to be used
  {18pt}% measure of space to leave above the theorem. E.g.: 3pt
  {12pt}% measure of space to leave below the theorem. E.g.: 3pt
  {}% name of font to use in the body of the theorem
  {}% measure of space to indent
  {\bfseries}% name of head font
  {:}% punctuation between head and body
  {2ex}% space after theorem head; " " = normal interword space
  {}% Manually specify head
\theoremstyle{homework} 

% Set up an Exercise environment and a Solution label.
\newtheorem*{exercisecore}{\@currentlabel}
\newenvironment{exercise}[1]
{\def\@currentlabel{#1}\exercisecore}
{\endexercisecore}

\newcommand{\localhead}[1]{\par\smallskip\noindent\textbf{#1}\nobreak\\}%
\newcommand\solution{\localhead{Solution:}}

% \newcommand{includematlab}[1]{\verbatiminput{#1}}

%%%%%%%%%%%%%%%%%%%%%%%%%%%%%%%%%%%%%%%%%%%%%%%%%%%%%%%%%%%%%%%%%%%%%%%%
%
% Stuff for getting the name/document date/title across the header
\makeatletter
\RequirePackage{fancyhdr}
\pagestyle{fancy}
\fancyfoot[C]{\ifnum \value{page} > 1\relax\thepage\fi}
\fancyhead[L]{\ifx\@doclabel\@empty\else\@doclabel\fi}
\fancyhead[C]{\ifx\@docdate\@empty\else\@docdate\fi}
\fancyhead[R]{\ifx\@docauthor\@empty\else\@docauthor\fi}
\headheight 15pt

\def\doclabel#1{\gdef\@doclabel{#1}}
\doclabel{Use {\tt\textbackslash doclabel\{MY LABEL\}}.}
\def\docdate#1{\gdef\@docdate{#1}}
\docdate{Use {\tt\textbackslash docdate\{MY DATE\}}.}
\def\docauthor#1{\gdef\@docauthor{#1}}
\docauthor{Use {\tt\textbackslash docauthor\{MY NAME\}}.}
\makeatother

% Shortcuts for blackboard bold number sets (reals, integers, etc.)
\newcommand{\Reals}{\ensuremath{\mathbb R}}
\newcommand{\Nats}{\ensuremath{\mathbb N}}
\newcommand{\Ints}{\ensuremath{\mathbb Z}}
\newcommand{\Rats}{\ensuremath{\mathbb Q}}
\newcommand{\Cplx}{\ensuremath{\mathbb C}}
%% Some equivalents that some people may prefer.
\let\RR\Reals
\let\NN\Nats
\let\II\Ints
\let\CC\Cplx

%%%%%%%%%%%%%%%%%%%%%%%%%%%%%%%%%%%%%%%%%%%%%%%%%%%%%%%%%%%%%%%%%%%%%%%%%%%%%%%%%%%%%%%
%%%%%%%%%%%%%%%%%%%%%%%%%%%%%%%%%%%%%%%%%%%%%%%%%%%%%%%%%%%%%%%%%%%%%%%%%%%%%%%%%%%%%%%
% 
% The main document start here.

% The following commands set up the material that appears in the header.
\doclabel{Math 426: Homework 1 (Part B)}
\docauthor{Andrew Player}
\docdate{August 28, 2020}

\newcommand{\vv}{\mathbf{v}}
\begin{document}

\def\vv{\mathbf{v}}
\def\vw{\mathbf{w}}

\begin{exercise}{Exercise \# 2}
\end{exercise}
\solution
\begin{verbatim}
Code:

  A = [10, -3; 4, 2];
  B = [1, 0; -1, 2];
  v = [1; 2];
  w = [1; 1];
  
  a = v'* w;
  b = v * w';
  c = A * v;
  d = A'* v;
  e = A * B;
  f = B * A;
  g = A * A;
  h = B\w;
  i = A\v;

Free Hand:

  I included the images with this submission. 
  They did not fit properly in this document.
  


















\end{verbatim}

\begin{exercise}{Exercise \# 4}
\end{exercise}
\solution
\begin{verbatim}
  x = [0, pi/6, 2*pi/6, 3*pi/6, 4*pi/6, 5*pi/6, pi, 
       7*pi/6, 8*pi/6, 9*pi/6, 10*pi/6, 11*pi/6, 2*pi];
  T = table(x, cos(x), sin(x));
\end{verbatim}


\begin{exercise}{Exercise \# 5}
\end{exercise}
\solution
\begin{verbatim}
a. 
  
  %  Plot function over large interval.
  subplot(2,1,1)
  x = [-6:.01:3];
  plot(x,2*cos(x)-exp(x))  
  title('plot of 2cos(x) - exp(x)')

  %  Zoom in on smaller interval about one root.
  subplot(2,1,2)
  xx = [-4.718:.0001:-4.716];
  plot(xx,2*cos(xx)-exp(xx))  
  axis([-4.718 -4.716 -0.02 0.02])
  title('zoomed view')
  Estimate: -4.717

b.

  Couldn't get this to plot... :(

  %  Plot function over large interval.
  subplot(2,1,1)
  x = [0:0.1:4];
  plot(x,(4*x.*sin(x) - 3)/(2+x.^2));
  axis([0 4 -2 2])

  %  Zoom in on smaller interval about one root.
  subplot(2,1,2)
  xx = [-4.718:.0001:-4.716];
  plot(xx,(4.*xx.*sin(xx) - 3)/(2+xx.^2))  
  axis([-4.718 -4.716 -0.02 0.02])


\end{verbatim}

\begin{exercise}{Exercise \# 7}
\end{exercise}
\solution
\begin{verbatim}
  theta = linspace(0, 2*pi, 1000);
  r = sqrt(2);
  x = 2 + r*cos(theta);
  y = 1 + r*sin(theta);
  plot(x, y)
  axis equal
  hold on
  theta2 = linspace(0, 2*pi, 1000);
  r2 = sqrt(3.5);
  x2 = 0 + r2*cos(theta2);
  y2 = 2.5 + r2*sin(theta2);
  plot(x2, y2)
  axis equal

  Intersections: 
  [1.869,  2.408]
  [0.6114,0.7319]
\end{verbatim}

\begin{exercise}{Exercise \# 9}
\end{exercise}
\solution
\begin{verbatim}
  square   = magic(5);
  row_sum  = sum(square, 2);
  col_sum  = sum(square, 1);
  diag_sum1 = sum(diag(square));
  diag_sum2 = sum(diag(flip(square)));
  is_magic = true;
  if diag_sum1 ~= diag_sum2
      is_magic = false;
  end
  for elem = row_sum
      if elem ~= diag_sum1
          is_magic = false;
          break;
      end
  end
  for elem = col_sum
      if elem ~= diag_sum1
          is_magic = false;
          break;
      end
  end
  if is_magic
      disp("Its Magic!")
  else
      disp("Its not Magic! :(")
  end
\end{verbatim}

\end{document}