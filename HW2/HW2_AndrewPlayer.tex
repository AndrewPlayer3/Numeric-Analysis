%%%%%%%%%%%%%%%%%%%%%%%%%%%%%%%%%%%%%%%%%%%%%%%%%%%%%%%%%%%%%%%%%%%%%%%%%%%%%%%%%%%%%%%
%%%%%%%%%%%%%%%%%%%%%%%%%%%%%%%%%%%%%%%%%%%%%%%%%%%%%%%%%%%%%%%%%%%%%%%%%%%%%%%%%%%%%%%
% 
% This top part of the document is called the 'preamble'.  Modify it with caution!
%
% The real document starts below where it says 'The main document starts here'.

\documentclass[12pt]{article}

\usepackage{amssymb,amsmath,amsthm}
\usepackage[top=1in, bottom=1in, left=1.25in, right=1.25in]{geometry}
\usepackage{fancyhdr}
\usepackage{graphicx}
\usepackage{enumerate}
\usepackage{verbatim}

% Comment the following line to use TeX's default font of Computer Modern.
\usepackage{times,txfonts}

\newtheoremstyle{homework}% name of the style to be used
  {18pt}% measure of space to leave above the theorem. E.g.: 3pt
  {12pt}% measure of space to leave below the theorem. E.g.: 3pt
  {}% name of font to use in the body of the theorem
  {}% measure of space to indent
  {\bfseries}% name of head font
  {:}% punctuation between head and body
  {2ex}% space after theorem head; " " = normal interword space
  {}% Manually specify head
\theoremstyle{homework} 

% Set up an Exercise environment and a Solution label.
\newtheorem*{exercisecore}{\@currentlabel}
\newenvironment{exercise}[1]
{\def\@currentlabel{#1}\exercisecore}
{\endexercisecore}

\newcommand{\localhead}[1]{\par\smallskip\noindent\textbf{#1}\nobreak\\}%
\newcommand\solution{\localhead{Solution:}}

% \newcommand{includematlab}[1]{\verbatiminput{#1}}

%%%%%%%%%%%%%%%%%%%%%%%%%%%%%%%%%%%%%%%%%%%%%%%%%%%%%%%%%%%%%%%%%%%%%%%%
%
% Stuff for getting the name/document date/title across the header
\makeatletter
\RequirePackage{fancyhdr}
\pagestyle{fancy}
\fancyfoot[C]{\ifnum \value{page} > 1\relax\thepage\fi}
\fancyhead[L]{\ifx\@doclabel\@empty\else\@doclabel\fi}
\fancyhead[C]{\ifx\@docdate\@empty\else\@docdate\fi}
\fancyhead[R]{\ifx\@docauthor\@empty\else\@docauthor\fi}
\headheight 15pt

\def\doclabel#1{\gdef\@doclabel{#1}}
\doclabel{Use {\tt\textbackslash doclabel\{MY LABEL\}}.}
\def\docdate#1{\gdef\@docdate{#1}}
\docdate{Use {\tt\textbackslash docdate\{MY DATE\}}.}
\def\docauthor#1{\gdef\@docauthor{#1}}
\docauthor{Use {\tt\textbackslash docauthor\{MY NAME\}}.}
\makeatother

% Shortcuts for blackboard bold number sets (reals, integers, etc.)
\newcommand{\Reals}{\ensuremath{\mathbb R}}
\newcommand{\Nats}{\ensuremath{\mathbb N}}
\newcommand{\Ints}{\ensuremath{\mathbb Z}}
\newcommand{\Rats}{\ensuremath{\mathbb Q}}
\newcommand{\Cplx}{\ensuremath{\mathbb C}}
%% Some equivalents that some people may prefer.
\let\RR\Reals
\let\NN\Nats
\let\II\Ints
\let\CC\Cplx

%%%%%%%%%%%%%%%%%%%%%%%%%%%%%%%%%%%%%%%%%%%%%%%%%%%%%%%%%%%%%%%%%%%%%%%%%%%%%%%%%%%%%%%
%%%%%%%%%%%%%%%%%%%%%%%%%%%%%%%%%%%%%%%%%%%%%%%%%%%%%%%%%%%%%%%%%%%%%%%%%%%%%%%%%%%%%%%
% 
% The main document start here.

% The following commands set up the material that appears in the header.
\doclabel{Math 426: Homework 2}
\docauthor{Andrew Player}
\docdate{September 4, 2020}

\newcommand{\vv}{\mathbf{v}}
\begin{document}

\def\vv{\mathbf{v}}
\def\vw{\mathbf{w}}

\begin{exercise}{Exercise \# 1}
\end{exercise}
\begin{verbatim}
function c=largest(a, b)
  if (a > b)
      c = a;
      return
  end
  c = b;
end
\end{verbatim} 

\begin{exercise}{Exercise \# 2}
\end{exercise}
\begin{verbatim}
function y=nextprime(x)
  while ~isprime(x)
    x = x + 1;
  end
  y = x;
end
\end{verbatim} 

\begin{exercise}{Exercise \# 3}
\end{exercise}
\begin{verbatim}
function x = buildseq(n)
  x = [1:n];
  for i = 2:n
    x(i) = (1/2)*x(i-1)+1;
  end
end
\end{verbatim} 

\begin{exercise}{Exercise \# 4}
\end{exercise}
\begin{verbatim}
function [x, history] = HW2bisect(f, a, b, delta)
    
  %Nx2 Array containing the intervals from each step 
  history = [];
  
  %Root approximation
  x = 0;
  
  %If the function does not have opposite signs along the interval
  %the function algorithm will not work so we throw an error and return
  if ~(f(a) > 0 && f(b) < 0) && ~(f(a) < 0 && f(b) > 0)
    disp("ERROR: Function does not have opposite signs at: ")
    disp([a,b])
    return
  end
  
  %Main loop for the algorithm
  while(true)
      
    %Bisection step
    c = (a + b) / 2;

    %Every iteration adds a row to history with the current interval
    history = [history ;[a, b]];

    %If f(x) == 0 or it is less then or equal to the tolerance
    %we have found the root and can return
    if f(c) == 0
        x = c;
        return
    elseif abs(f(c)) <= delta
        x = c;
        return
    end

    %Maintain opposite signs
    if sign(f(a)) == sign(f(c))
        a = c;
    else
        b = c;
    end
  end
end
\end{verbatim} 

\begin{exercise}{Chapter 4 \# 2(a)}
\end{exercise}
\begin{verbatim}
\end{verbatim} 

\begin{exercise}{Chapter 4 \# 5}
\end{exercise}
\begin{verbatim}
\end{verbatim} 

\end{document}