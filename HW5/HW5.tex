%%%%%%%%%%%%%%%%%%%%%%%%%%%%%%%%%%%%%%%%%%%%%%%%%%%%%%%%%%%%%%%%%%%%%%%%%%%%%%%%%%%%%%%
%%%%%%%%%%%%%%%%%%%%%%%%%%%%%%%%%%%%%%%%%%%%%%%%%%%%%%%%%%%%%%%%%%%%%%%%%%%%%%%%%%%%%%%
% 
% This top part of the document is called the 'preamble'.  Modify it with caution!
%
% The real document starts below where it says 'The main document starts here'.

\documentclass[12pt]{article}

\usepackage{amssymb,amsmath,amsthm}
\usepackage[top=1in, bottom=1in, left=1.25in, right=1.25in]{geometry}
\usepackage{fancyhdr}
\usepackage{graphicx}
\usepackage{enumerate}
\usepackage{verbatim}

% Comment the following line to use TeX's default font of Computer Modern.
\usepackage{times,txfonts}

\newtheoremstyle{homework}% name of the style to be used
  {18pt}% measure of space to leave above the theorem. E.g.: 3pt
  {12pt}% measure of space to leave below the theorem. E.g.: 3pt
  {}% name of font to use in the body of the theorem
  {}% measure of space to indent
  {\bfseries}% name of head font
  {:}% punctuation between head and body
  {2ex}% space after theorem head; " " = normal interword space
  {}% Manually specify head
\theoremstyle{homework} 

% Set up an Exercise environment and a Solution label.
\newtheorem*{exercisecore}{\@currentlabel}
\newenvironment{exercise}[1]
{\def\@currentlabel{#1}\exercisecore}
{\endexercisecore}

\newcommand{\localhead}[1]{\par\smallskip\noindent\textbf{#1}\nobreak\\}%
\newcommand\solution{\localhead{Solution:}}



% \newcommand{includematlab}[1]{\verbatiminput{#1}}

%%%%%%%%%%%%%%%%%%%%%%%%%%%%%%%%%%%%%%%%%%%%%%%%%%%%%%%%%%%%%%%%%%%%%%%%
%
% Stuff for getting the name/document date/title across the header
\makeatletter
\RequirePackage{fancyhdr}
\pagestyle{fancy}
\fancyfoot[C]{\ifnum \value{page} > 1\relax\thepage\fi}
\fancyhead[L]{\ifx\@doclabel\@empty\else\@doclabel\fi}
\fancyhead[C]{\ifx\@docdate\@empty\else\@docdate\fi}
\fancyhead[R]{\ifx\@docauthor\@empty\else\@docauthor\fi}
\headheight 15pt

\def\doclabel#1{\gdef\@doclabel{#1}}
\doclabel{Use {\tt\textbackslash doclabel\{MY LABEL\}}.}
\def\docdate#1{\gdef\@docdate{#1}}
\docdate{Use {\tt\textbackslash docdate\{MY DATE\}}.}
\def\docauthor#1{\gdef\@docauthor{#1}}
\docauthor{Use {\tt\textbackslash docauthor\{MY NAME\}}.}
\makeatother

%% General formatting parameters
\parindent 0pt
\parskip 12pt plus 1pt

% Shortcuts for blackboard bold number sets (reals, integers, etc.)
\newcommand{\Reals}{\ensuremath{\mathbb R}}
\newcommand{\Nats}{\ensuremath{\mathbb N}}
\newcommand{\Ints}{\ensuremath{\mathbb Z}}
\newcommand{\Rats}{\ensuremath{\mathbb Q}}
\newcommand{\Cplx}{\ensuremath{\mathbb C}}
%% Some equivalents that some people may prefer.
\let\RR\Reals
\let\NN\Nats
\let\II\Ints
\let\CC\Cplx

%%%%%%%%%%%%%%%%%%%%%%%%%%%%%%%%%%%%%%%%%%%%%%%%%%%%%%%%%%%%%%%%%%%%%%%%%%%%%%%%%%%%%%%
%%%%%%%%%%%%%%%%%%%%%%%%%%%%%%%%%%%%%%%%%%%%%%%%%%%%%%%%%%%%%%%%%%%%%%%%%%%%%%%%%%%%%%%
% 
% The main document start here.

% The following commands set up the material that appears in the header.
\doclabel{Math 426: Homework 4}
\docauthor{Andrew Player}
\docdate{September 23, 2020}

\newcommand{\vv}{\mathbf{v}}
\begin{document}

\begin{exercise} Problem 5.1 [Read the bulleted list describing rounding modes on page 116 first.]
\end{exercise}
\begin{verbatim}
a. Rounds to 2.
0 10000000000 0000000000000000000000000000000000000000000000000000
b. Rounds to 5.
0 10000000001 0100000000000000000000000000000000000000000000000000 
c. Rounds to -5.
1 10000000001 0100000000000000000000000000000000000000000000000000
d. Rounds to -6
1 10000000001 1000000000000000000000000000000000000000000000000000
\end{verbatim}

\begin{exercise} Problem 5.2  For full credit you should correctly justify via a hand computaiton how you arrived at this answer.
\end{exercise}
\begin{verbatim}
Rounding 50.2 to the nearest gives 50.
The highest power of 2 lower than 50 is 32, which is 2^5.
Also, the sign is 0. So the mantissa must be 50/32 = 1.5625.
So, the IEEE representation is:
0 2^5         1.5625
0 10000000100 1001000000000000000000000000000000000000000000000000 
\end{verbatim}

\begin{exercise} Problem 5.3 
\end{exercise}
\begin{verbatim}
eps(2) = 4.440892098500626e-16
\end{verbatim}

\begin{exercise} Problem 5.4
\end{exercise}
\begin{verbatim}
eps(201) = 2.842170943040401e-14
\end{verbatim}

\begin{exercise} Problem 5.5
\end{exercise}
There are 1023 values for the exponent. Then there are 52 bits
for the mantissa that have two possible values, so $2^{52}$ possible
values for each of the 1023 exponents. Then, there is a negative 
and positive possible value. Therefore, there is a total of $2^{53}*(2^{10}-1)$ 
possible values.

\begin{exercise} Problem 5.8
\end{exercise}
\begin{verbatim}
Epsilon:  2^-3
Smallest: 0 1.000 -1 = 0.5
Largest:  0 1.111  1 = 3.75
\end{verbatim}

\begin{exercise} Problem 5.9 [Wait until after Friday to start]
\end{exercise}
a. Floating point addition between two numbers is commutative.
So a + b and b + a should always be equivalent since they should
always be rounded the same. However, this would not be true for three numbers.
For example, 0.1 + 0.2 + 0.3 != 0.3 + 0.2 + 0.1, in matlab, since this 
introduces different intermediate values on both sides that will be rounded
to different values.
\newline
\newline
b. It does not hold true.
\newline
Matlab Counter-Example:
\begin{verbatim}
>> (0.1 + 0.2)  + 0.3 == 0.1 + (0.2 + 0.3)

ans =

  logical

    0
\end{verbatim}
c. 
\newline
\newline
Relative Error:
\newline
$\frac{|\frac{ab(1+d_1)}{c}(1+d_2)-\frac{ab}{c}|}{|\frac{ab}{c}|}$
\newline   
$\frac{|\frac{ab}{c}((1+d_1)(1+d_2)-1)|}{\frac{ab}{c}}$
\newline
$|((1+d_1)(1+d_2)-1)|$
\newline
$|d_1 + d_2 + d_1d_2|$
\newline
\newline
Additionally, If c = 0, the possible values are Inf, NaN, and -Inf.


\begin{exercise} Problem 5.15
\end{exercise}
a. $\frac{9999990463256835938}{10000000000000000000}$
\newline
b. $Error = 0.1 - 0.09999990463256835938 = 9.536743164062 * 10^{-8}$
\newline
c. $0.343322753906232$
\newline
d. $3750mph = \frac{3750}{60^2}miles/s$
\newline
   $1.041666666666667 miles$
\end{document}
